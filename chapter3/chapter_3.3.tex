\documentclass{report}

\input{~/latex/template/preamble.tex}
\input{~/latex/template/macros.tex}

\title{\Huge{Chapter 3.3 Lecture Notes}}
\author{\huge{Matt Warner}}
\pagestyle{fancy}
\fancyhf{}
\rhead{3.3 COMPLEX ZEROS AND THE FUNDAMENTAL THEOREM OF ALGEBRA}
\lhead{\leftmark}
\cfoot{\thepage}
% \usepackage[default]{sourcecodepro}
% \usepackage[T1]{fontenc}

\pgfpagesdeclarelayout{boxed}
{
  \edef\pgfpageoptionborder{0pt}
}
{
  \pgfpagesphysicalpageoptions
  {%
    logical pages=1,%
  }
  \pgfpageslogicalpageoptions{1}
  {
    border code=\pgfsetlinewidth{1.5pt}\pgfstroke,%
    border shrink=\pgfpageoptionborder,%
    resized width=.95\pgfphysicalwidth,%
    resized height=.95\pgfphysicalheight,%
    center=\pgfpoint{.5\pgfphysicalwidth}{.5\pgfphysicalheight}%
  }%
}

\pgfpagesuselayout{boxed}

\begin{document}

  \maketitle

  \begin{huge}
    \noindent \textbf{Complex Zeros}
  \end{huge}

  \bigbreak \bigbreak 

  \begin{large}
   \noindent \textbf{Fundamental Theorem of Algebra}
  \end{large}

  \bigbreak

  \noindent If a polynomial $f(x)$ has positive degrees and complex coefficients, then $f(x)$ has a least one complex zero.
  \bigbreak \bigbreak
  \begin{large}

    \noindent \textbf{Complete Factorization Theorem for polynomials}

  \end{large}

  \bigbreak
  \noindent if $f(x)$ is a polynomial of degree $n>0$, then there exists $n$ complex numbers $c_1,c_2,...,c_n$ such that $f(x)=a(x-c_1)(x-c_2)...(x-c_n)$ where $a$ is the leading coefficient of $f(x)$. Each number $c_k$ is a zero of $f(x)$.
  \bigbreak \bigbreak
  \begin{large}
    \noindent \textbf{Conjugate Pairs theorem}
  \end{large}
  \bigbreak
  \noindent Let $f(x)$ be a polynomial whose coefficients are real numbers. if $r=a+bi$ is a zero of $f$, then the Conjugate $\bar{r} = a-bi$ is also a zero of $f$
  \bigbreak \bigbreak
  \begin{large}
  \noindent \underline{Example 1}
  find $f(x)$ given the zeros.
    
  \end{large}

  \bigbreak
  \noindent a) -3, 1, -7$i$: degree 3
  \pf{Solution}     
  \bigbreak
  1) $f(x)=(x+3)(x-1+7i)(x-1-7i)$
  \bigbreak
  2) $f(x)=(x+3)((x-1)^2(-49i^2))$
  \bigbreak
  3) $f(x) = (x+3)(x-2x+1 + 49)$
  \bigbreak
  4)\textbf{$f(x)=x^3+x^2+44x+150$}
  \bigbreak \bigbreak
  \noindent b) 0, 3i, 4+i
  \pf{Solution}
  \bigbreak
  1) $f(x)= x(x-3i)(x+3i)(x-4+i)(x-4-i)$
  \bigbreak
  2) $f(x)= x(x^2-9i^2)((x-4)^2-i^2)$
  \bigbreak
  3) $f(x)= (x^3+9x)(x^2-8x+17)$
  \bigbreak
  4) $f(x)= x^5-8x^4+26x^3-72x^2+153x$
  \bigbreak \bigbreak
  \begin{large}
    \noindent \underline{Example 2}
    Use the given zero to find the remaining zeros of the function
    
  \end{large}
  \bigbreak
  \noindent a) $f(x)= x^3+3x^2+25x+65$
  \bigbreak
  \noindent Zero: -5i
  \pf{Solution} 
  \bigbreak
   1) Zeros:(x+5i)(x-5i)(x-c)
   \bigbreak
   2) $\frac{f(x)}{(x+5i)(x-5i)}$
   \bigbreak
   3) $\frac{f(x)}{x^2-25i^2}$
  \bigbreak 
   =$\frac{f(x)}{x^2+25}$

   \bigbreak
   4) Using polynomial long division we get: $(x+3)$ as our $3^{rd}$ zero
   \bigbreak \bigbreak
   \noindent b) $f(x)=3x^4+5x^3+25x^2+45x-18$; Zero: 3i
  \pf{Solution}
  \bigbreak
  1) \large Known zeros are: $(3i, -3i)$ 
  \bigbreak
  2) \large$\frac{f(x)}{(x+3i)(x-3i)}$
  \bigbreak
  3) \large$\frac{f(x)}{(x^2+9)}$
  \bigbreak
  4) Using polynomial long division  we get: $\frac{1}{3}$, -2 as the remaining zeros 
  \bigbreak \bigbreak
  \begin{large}
    \noindent \underline{Example 3}
    Find all complex zeros of $f(x)$ 

  \end{large}
  \bigbreak \noindent \bigbreak \noindent
  \pf{Solution}

  $$f(x)=x^3+13x^2+57x+85$$
  \bigbreak
  $$\frac{1,85,17,5}{1}$$
  \bigbreak
  $$\pm{1,85,17,5}$$
  \bigbreak
  \begin{center}
  \textbf{Using long division we get -5 as our first zero}
  \bigbreak
  \textbf{Using the quadratic equation with our remainder we get $-4\pm{i}$ as our remaining zeros}
  \end{center}
  








\end{document}
