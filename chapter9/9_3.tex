\documentclass{report}

\input{~/latex/template/preamble.tex}
\input{~/latex/template/macros.tex}

\title{\Huge{Section 9.3 Notes}}
\author{\huge{Matt Warner}}
\date{\huge{}}
\pagestyle{fancy}
\fancyhf{}
\rhead{Chapter 9 Section 3 Notes}
\lhead{\leftmark}
\cfoot{\thepage}
% \usepackage[default]{sourcecodepro}
% \usepackage[T1]{fontenc}
\usepackage{graphicx}
\usepackage{pgfplots}
\pgfplotsset{compat=1.18}
\graphicspath{ {/home/mattwarner/school/pictures/} }
\pgfpagesdeclarelayout{boxed}
{
  \edef\pgfpageoptionborder{0pt}
}
{
  \pgfpagesphysicalpageoptions
  {%
    logical pages=1,%
  }
  \pgfpageslogicalpageoptions{1}
  {
    border code=\pgfsetlinewidth{1.5pt}\pgfstroke,%
    border shrink=\pgfpageoptionborder,%
    resized width=.95\pgfphysicalwidth,%
    resized height=.95\pgfphysicalheight,%
    center=\pgfpoint{.5\pgfphysicalwidth}{.5\pgfphysicalheight}%
  }%
}

\pgfpagesuselayout{boxed}

\begin{document}
	\maketitle
\begin{LARGE}
	\begin{center}
		\textbf{9.3 THE ELLIPSE}
	\end{center}
\end{LARGE}
\bigbreak \noindent \bigbreak \noindent

\begin{large}
	\noindent \textbf{\underline{Definition}}
\end{large}
\bigbreak
\noindent An \textbf{ellipse} is the set of all points in a plane, the sum of whose distances from two fixed points (the \textbf{foci}) in the plane is a positive constant. The foci are a distance $c$ from the center, where $c^2=a^2-b^2$.
\vspace{5mm}

\noindent The \textbf{major axis} of the ellipse is the longest line segment passing through the center and foci. The end points of the major axis are called the vertices of the ellipse. Vertices are a distance of $a$ from the center.
\vspace{5mm}

\noindent The \textbf{minor axis} of the ellipse is the shortest line segment passing through the center. \\ The length of the major axis is $2a$, and the length of the minor axis is $2b$.
\bigbreak \bigbreak
\begin{center}
{Standard Equation of an ellipse with center $(h, k)$ and $c^2=a^2-b^2$}
\end{center}
\begin{center}
  
\begin{tabular}{|l|l|}
\hline Standard equation, foci, vertices & Standard equation, foci, vertices \\
\hline$\frac{(x-h)^2}{a^2}+\frac{(y-k)^2}{b^2}=1$ where $a>b>0$ & $\frac{(x-h)^2}{b^2}+\frac{(y-k)^2}{a^2}=1$ where $a>b>0$ \\
$\begin{array}{l}\text { Major axis: parallel to the } x \text {-axis } \\
\text { Foci: } F(h \pm c, k) \\
\text { Vertices: }(h \pm a, k)\end{array}$ & $\begin{array}{l}\text { Major axis: parallel to the } y \text {-axis } \\
\text { Foci: } F(h, k \pm c) \\
\text { Vertices: }(h, k \pm a)\end{array}$ \\
\hline
\end{tabular}
\end{center}
\bigbreak \noindent \bigbreak \noindent
\vspace{7mm}

\begin{large}
	\noindent \textbf{Example 1:}  
	Sketch the graph of the functions. Find the vertices and foci.
\end{large}
\bigbreak
\vspace{3mm}

\begin{large}
\noindent a) $y^2 +9x^2 = 9$ 
\end{large}
\vspace{3mm}
\begin{center}
	$$\frac{y^2+9x^2}{9}=\frac{9}{9}$$
$$x^2+\left(\frac{y}{3}\right)^2 = 1$$

 Center: $(0,0)$

 $a=3$

 $b=1$
\vspace{3mm}

 \nt{a is always larger than $b$, so $a=3$}
 \vspace{2mm}

Now, we use the formula, $c^2 = a^2 - b^2$ to find c. 
so, $c=\sqrt{8} \rightarrow 2\sqrt{2}$
\vspace{2mm}

	vertices = $(0, \pm{3})$
	\vspace{2mm}

	Foci = $(0,\pm2\sqrt{2})$

\end{center}

\vspace{5mm}
\hline 
\bigbreak \noindent \bigbreak \noindent
\begin{large}
	b) $\frac{(x+3)^2}{9}+\frac{(y-2)^2}{4}=1$
\end{large}
\bigbreak 
Center = (-3,2)
\vspace{2mm}

 $a =3$
\vspace{2mm}

 $b=2$

 $$c^2=3^2-2^2 \rightarrow c=\sqrt{5}$$
 \vspace{3mm}

 \begin{center}
   
\includegraphics[scale=.7] {graph1}
 \end{center}
\vspace{3mm}

\noindent As you can see in the figure, the center is located at $(-3,2)$, and we can create the ellipse by addding $\pm{2}$ to the y-axis at the center, and adding $\pm{-3}$ to the x-axis at our center point. (These are our $a$ and $b$ values.)
\vspace{2mm}

Thus, The vertices are listed as $\rightarrow$ $ (-6,2), (0,2)$
\vspace{2mm}

Foci = $(-3\pm{\sqrt{5}}, 2)$, (x-value in the center point $\pm{c}$, y-value of the center point.)
\bigbreak \noindent \bigbreak \noindent
\noindent
\begin{large}
 c) $9 x^2+4 y^2-18 x+16 y-11=0$
\end{large}
\vspace{5mm}

\vspace{3mm}
\nt{Compleating the Sqaure: \\ $\left(\frac{b}{2}\right)^2$ = new value for c in the equation}
\vspace{3mm}

$$
9\left(x^2-2 x+1\right)+4\left(y^2+4 y+4\right)=11+9+16
$$
$$
\frac{(x-1)^2}{36}+\frac{4(y+2)^2}{36}=\frac{36}{36}
$$
$$
\frac{(x-1)^2}{4}+\frac{(y+2)^2}{9}=1
$$
\vspace{4mm}

Center = $(1,-2)$

$a = 3$

$b = 2$

$c = \sqrt{5}$







\end{document}
