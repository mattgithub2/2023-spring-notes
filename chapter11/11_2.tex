\documentclass{report}

\input{~/latex/template/preamble.tex}
\input{~/latex/template/macros.tex}

\title{\Huge{Chapter 11.2 Notes}}
\author{\huge{Matt Warner}}
\date{\huge{}}
\pagestyle{fancy}
\fancyhf{}
\rhead{Section 11.2 - Arithmetic Sequence}
\lhead{\leftmark}
\cfoot{\thepage}
% \usepackage[default]{sourcecodepro}
% \usepackage[T1]{fontenc}

\usepackage{tikz}
\usepackage{pgfplots}
\pgfplotsset{compat=1.18}

\pgfpagesdeclarelayout{boxed}
{
  \edef\pgfpageoptionborder{0pt}
}
{
  \pgfpagesphysicalpageoptions
  {%
    logical pages=1,%
  }
  \pgfpageslogicalpageoptions{1}
  {
    border code=\pgfsetlinewidth{1.5pt}\pgfstroke,%
    border shrink=\pgfpageoptionborder,%
    resized width=.95\pgfphysicalwidth,%
    resized height=.95\pgfphysicalheight,%
    center=\pgfpoint{.5\pgfphysicalwidth}{.5\pgfphysicalheight}%
  }%
}

\pgfpagesuselayout{boxed}

\begin{document}
	\maketitle
\begin{LARGE}
	\begin{center}
		\textbf{ARITHMETIC SEQUENCE} 
	\end{center}
\end{LARGE}
\bigbreak \noindent \bigbreak \noindent
\begin{large}
 Determine Whether a Sequence is Arithmetic 
 \vspace{5mm}

 $$a_1 = a, \hspace{10mm} a_n = a_{n-1} +d$$
\end{large}
\bigbreak \noindent

\nt{
	In the equation above, the variable $d$, is refered to as the \textit{common difference}, the difference between consecutive terms in the sequence.

	 $$d = a_n - a_{n-1}$$
}
\bigbreak \noindent \bigbreak \noindent \bigbreak \noindent 

\begin{large}
	  \noindent \textbf{Example 1} 
	 Show that the sequence is arithmetic. List the first term and the common difference.
\end{large}
\bigbreak \noindent

\line(1,0){450}

\begin{center}
\begin{large}
	{\LARGE{a)}} $ 4, 2,0, -2,\ldots$
\end{large}
\end{center}

\line(1,0){450}
\bigbreak \noindent

\begin{center}
 $d = (2-4) \rightarrow (0-2) \rightarrow (-2-0)$
 \vspace{2mm}

 $d = -2$
 \vspace{4mm}

 The common difference shows that this sequence is arithmetic, not geometric.
\end{center}
\bigbreak \noindent \bigbreak \noindent

\line(1,0){450}

\begin{center}
\begin{large}
	{\LARGE{b)}} $\left\{s_n\right\}=\{4 n-1\}$
\end{large}
\end{center}

\line(1,0){450}

\begin{center}
 $s_1 = 4(1) - 1$
 $=3$
\vspace{2mm}

$s_2 = 4(2) - 1$
$ = 7$
\vspace{2mm}

$s_3 = 4(3) -1$
$=11$
\vspace{2mm}

$s_4 = 4(4) - 1$
$=15$
\vspace{2mm}

Here we can see that the common difference between the terms in the sequence is {\large{4}}.
\end{center}
\vspace{3mm}

Alternatively,

$$
\begin{aligned}
S_n-S_{n-1} & =4 n-1-(4(n-1)-1) \\
& =4 n-1-[4 n-4-1] \\
& =4 n-1-[4 n-5] \\
& =4 n-1-4 n+5 \\
& =4 \quad \text { common differnece}
\end{aligned}
$$
\newpage
\noindent
\begin{LARGE}
	\textbf{Find a Formula for an Arithmetic Sequence}  
\end{LARGE}
\bigbreak \noindent \bigbreak \noindent
\begin{large}
 $n$th Term of an Arithmetic Sequence 
 \vspace{4mm}

\begin{mdframed}
\noindent For an arithmetic sequence, $\{a_n\}$ whose first term is $a$ and whose common difference is $d$, the $n$th term is determined by the formula
 $$a_n = a + (n-1)d$$
\end{mdframed}
\end{large}
\nt{
$$
\begin{aligned}
& a_1=a \\
& a_2=a_1+d=a+d \\
& a_3=a_2+d=a+d+d=a+2 d \\
& a_4=a_3+d=a+2 d+d=a+3 d \\
& \vdots \\
& a_n=a_{n-1}+d= a +(n-1)d
\end{aligned}
$$
}
\bigbreak \noindent \bigbreak \noindent
\begin{large}
 \begin{center}
	 \textbf{Example 2} 
	 Find the twenty fourth term of the arithmetic sequence:
 \end{center} 
\end{large}

\line(1,0){450}

\begin{center}
\begin{large}
	$-3,0,3,6,\ldots$
\end{large}
\end{center}

\line(1,0){450}

\begin{center}
	$a = -3$ \hspace{10mm} $d = 3$
	\vspace{4mm}

	$a_n = a +(n-1)d$ 
	\vspace{2mm}

	$-3 +(n-1)\cdot 3$
	\vspace{2mm}

	$-3+3n-3 \rightarrow 3n-6$
	\vspace{2mm}
	
	$a_{24} = 3(24) -6$
	\vspace{2mm}

	$=\boxed{66}$
\end{center}
\bigbreak \noindent \bigbreak \noindent \bigbreak \noindent

\begin{large}
\noindent \textbf{Example 3:} 
	 The sixth term of an arithmetic sequence is 26, and the nineteenth term is 78. Find the first term and the common difference. Give a recursive formula for the sequence. What is the nth term of the sequence? 
\end{large}

\begin{center}
\hspace{10mm} $a_6 = 26$ \hspace{10mm} $a_{19} = 78$ \hspace{10mm}
\vspace{3mm}

$a_n = a + (n-1)d$
$$
\begin{aligned}
& 26=a+(6-1) d \\
& 78=a+(19-1) d
	\end{aligned} \hspace{2mm} \Rightarrow\left\{\begin{array}{l}
26=a+5 d \\
78=a+18 d .
\end{array}\right.
$$
$$52 = 13d$$
$$4 =d$$
\end{center}
So,

$26 = a + 5(4) =6$

$a_1 = 6$
\newpage
\noindent
\begin{LARGE}
	\textbf{Sum of n Terms of an Arithmetic Sequence}
\end{LARGE}
\bigbreak \noindent
\begin{large}
	Let $\{a_n\}$ be an arithmetic sequence with first term $a$ and common difference $d$. The sum $s_n$ of the first n terms of $\{a_n\}$ is 
	\vspace{3mm}

	$$S_n=\frac{n}{2}[2 a+(n-1) d]=\frac{n}{2}\left(a+a_n\right)$$
\end{large}
\bigbreak \noindent \bigbreak \noindent \bigbreak \noindent
\begin{large}
\begin{center}
	\textbf{Find the sum of the first n terms of the sequence} 
\end{center}  
\end{large}

\line(1,0){450}

\begin{center}
\begin{large}
	$\{4n + 2\}$
\end{large}
\end{center}

\line(1,0){450}
\bigbreak \noindent

\begin{minipage}[t]{0.45\textwidth}
    \begin{align*}
        a_1 &= 4(1) + 2 = 6 \\
        a_2 &= 10 \\
        a_3 &= 14 \\
        a_4 &= 18 \\
		a_5 &= 22 \\
		a_6 &= 26 \\
		a_7 &= 30 \\
    \end{align*}
\end{minipage}\hfill%
\begin{minipage}[t]{0.45\textwidth}
    \begin{align*}
        s_n &= 6+10+14+18+\ldots+4n+2 \\
        &= \frac{n}{2}(a_1+a_n) \\
        &= \frac{n}{2}(8+4n) \\
        &= \frac{4n(2+n)}{2} \\
		&= \boxed{2n(2+n)} \\
    \end{align*}
\end{minipage}
\bigbreak \noindent \bigbreak \noindent \bigbreak \noindent

\begin{large}
	\noindent \textbf{Example 5:}
	The conrner section of a stadium has 20 seats in the first row and 40 rows in all. Each successive row contains two additional seats. How many seats are in this section?
\end{large}
\vspace{4mm}

\pf{Solution}

Given that each row has 2 more seats then the previous rows.
$$s_{40} = 20 +22+24+\ldots$$
$$a_n = a +(n-1)d$$
$$ = 20 +39 \cdot 2$$
$$a_n = 98$$
$$s_{40} = \frac{n}{2}(a+a_n)$$
$$\frac{40}{2}(20+98)$$

$$ = \boxed{2360}$$
\end{document}
