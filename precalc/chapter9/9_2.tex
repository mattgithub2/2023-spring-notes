\documentclass{report}

\input{~/latex/template/preamble.tex}
\input{~/latex/template/macros.tex}

\title{\Huge{9.2 Notes}}
\author{\huge{Matt Warner}}
\date{\huge{}}
\pagestyle{fancy}
\fancyhf{}
\rhead{Chapter 9.2 - The Parabola}
\lhead{\leftmark}
\cfoot{\thepage}
% \usepackage[default]{sourcecodepro}
% \usepackage[T1]{fontenc}

\usepackage{pgfplots}
\usepackage{tikz}
\pgfplotsset{compat=1.18}
\pgfpagesdeclarelayout{boxed}
{
  \edef\pgfpageoptionborder{0pt}
}
{
  \pgfpagesphysicalpageoptions
  {%
    logical pages=1,%
  }
  \pgfpageslogicalpageoptions{1}
  {
    border code=\pgfsetlinewidth{1.5pt}\pgfstroke,%
    border shrink=\pgfpageoptionborder,%
    resized width=.95\pgfphysicalwidth,%
    resized height=.95\pgfphysicalheight,%
    center=\pgfpoint{.5\pgfphysicalwidth}{.5\pgfphysicalheight}%
  }%
}

\pgfpagesuselayout{boxed}

\begin{document}
	\maketitle
	

\begin{LARGE}
	\begin{center}
		\textbf{THE PARABOLA} 
	\end{center}
\end{LARGE}
\bigbreak \noindent
\begin{large}

\noindent A \textbf{parabola} is the set of all points in a plane equidistant from a fixed point $F$ (the \textbf{focus} and a fixed line $l$ (th \textbf{directrix} that lie in the plane.
\vspace{5mm}

\noindent The \textbf{axis} of the parabola is the line through $F$ that is perpendicular to the directrix.
\vspace{5mm}

\noindent The \textbf{vertex} of the parabola is the point $V$ on the axis halfway from $F$ to $l$.

\end{large}
\bigbreak \noindent

\begin{center}
\begin{align*}
&\text { Parabola with Vertex } V(h, k)\\
&\begin{array}{|l|l|l|}
\hline \multicolumn{1}{|c|}{\begin{array}{l}
\text { Standard equation, focus, } \\
\text { directrix }
\end{array}} & \text { Graph for } p>0 & \text { Graph for } p<0 \\
\hline \quad(x-h)^2=4 p(y-k) & & \\
\begin{array}{l}
\text { Focus: } F(h, k+p) \\
\text { Directrix: } y=k-p \\
\text { Length of latus rectum: } 4 p
\end{array} & & \\
\hline(y-k)^2=4 p(x-h) & & \\
\text { Focus: } F(h+p, k) \\
\text { Directrix: } x=h-p \\
\text { Length of latus rectum: } 4 p & & \\
\hline
\end{array}
\end{align*}
\end{center}

\bigbreak \noindent \bigbreak \noindent
\begin{large}
	\textbf{Ex 1}
	Sketch the graph of the following functions. Find the vertex, focus, and directrix
\end{large}
\bigbreak \noindent

$$
\begin{array}{|c|c|c|c|}
\hline \text { Vertex } & \text { Focus } & \text { Directrix } & \text { Equation } \\
\hline(0,0) & (a, 0) & x=-a & y^2=4 a x \\
\hline(0,0) & (-a, 0) & x=a & y^2=-4 a x \\
\hline(0,0) & (0, a) & y=-a & x^2=4 a y \\
\hline(0,0) & (0,-a) & y=a & x^2=-4 a y \\
\hline
\end{array}
$$
\bigbreak \noindent \bigbreak \noindent

\begin{large}
\noindent a) $x^2 = -3y$ 
\end{large}
\vspace{3mm}

$$4p = -3$$
$$p = \frac{-3}{4}$$
Vertex: $(0,0)$, 
\vspace{2mm}
\noindent Focus: $(0, \frac{-3}{4})$
\begin{center}
		\begin{center}
			\begin{tikzpicture}
				\begin{axis}[restrict y to domain=-10:10, restrict x to domain=-10:10,samples=200, xmin=-10, xmax=10, ymin=-10, ymax=10, axis lines=middle]
					\addplot[color=red] {-(1/3)*x^2};
				\end{axis} 
			\end{tikzpicture}
		\end{center}
\end{center}
\bigbreak \noindent \bigbreak \noindent
\begin{large}
	b) $\left(y+1\right) = -12\left(x+2\right)$ 

\end{large}

\begin{center}
Vertex = $(-2,-1)$
\vspace{2mm}

$4p = -12$  $\rightarrow$ focal width = 12 $\rightarrow$ split in half: 6
\vspace{2mm}

$p = -3$
\vspace{2mm}

Focus: $(-5,-1)$ $\rightarrow$ (The focus is p, or -3 away from the vertex.) (x-axis)
\vspace{2mm}

Focus: $(-5,5)$ $\rightarrow$ (This focus is 6 units from our vertex) (y-axis)
\vspace{2mm}

Thus,
\bigbreak \noindent \bigbreak \noindent \bigbreak \noindent

\begin{tikzpicture}
\begin{axis}[
  axis lines=middle,
  xlabel=$x$,
  ylabel=$y$,
  xmin=-5, xmax=5,
  ymin=-8, ymax=5,
  domain=-10:2,
  samples=100
  ]
  \addplot [blue, thick] {sqrt(-12*(x+2))-1};
  \addplot [blue, thick] {-sqrt(-12*(x+2))-1};
\end{axis}
\end{tikzpicture}
\end{center}
\bigbreak \noindent \bigbreak \noindent
\begin{large}
	\textbf{Ex 2:}
	Find the equation of the parabola that satisfies the given conditions.
\end{large}
\bigbreak \noindent \bigbreak \noindent \bigbreak \noindent
\begin{large}
 a) Focus $F$$(-3,-2)$, directrix $y=1$ 
\end{large}
\vspace{3mm}

\begin{center}
Vertex: $(-3,\frac{-1}{2})$
\vspace{2mm}

$(x+3)^2 = 4p(y+\frac{1}{2})$
\vspace{2mm}

$(x+3)^2=4(-1.5)\left(y+\frac{1}{2}\right)$
\vspace{2mm}

$(x+3)^2=-6\left(y+\frac{1}{2}\right)$
\end{center}
\bigbreak \noindent \bigbreak \noindent
\begin{large}
b) Vertex $V(3,-2)$, axis (axis of symmetry) parallel  to the x-axis, and y-intercept 1.
\vspace{2mm}


\end{large}
\bigbreak \noindent
\begin{center}
 $h = 3$ 
 \vspace{2mm}

 $k =-2$ 
 \vspace{2mm}
 
$(y-k)^2=4 p(x-h)$
\vspace{2mm}

$(y+2)^2=4 p(x-3)$
\vspace{2mm}

$(1+2)^2=4 p(0-3)$
\vspace{2mm}

$-3 = 4p$
\vspace{2mm}

\boxed{$(y+2)^2=-3(x-3)$}
\end{center}










\end{document}
