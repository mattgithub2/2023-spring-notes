\documentclass{report}

\input{~/latex/template/preamble.tex}
\input{~/latex/template/macros.tex}

\title{\Huge{Section 4.3 Notes}}
\author{\huge{Matt Warner}}
\date{\huge{}}
\pagestyle{fancy}
\fancyhf{}
\rhead{4.3 EXPONENTIAL FUNCTIONS}
\lhead{\leftmark}
\cfoot{\thepage}
% \usepackage[default]{sourcecodepro}
% \usepackage[T1]{fontenc}

\usepackage{pgfplots}
\pgfplotsset{compat=1.18}

\pgfpagesdeclarelayout{boxed}
{
  \edef\pgfpageoptionborder{0pt}
}
{
  \pgfpagesphysicalpageoptions
  {%
    logical pages=1,%
  }
  \pgfpageslogicalpageoptions{1}
  {
    border code=\pgfsetlinewidth{1.5pt}\pgfstroke,%
    border shrink=\pgfpageoptionborder,%
    resized width=.95\pgfphysicalwidth,%
    resized height=.95\pgfphysicalheight,%
    center=\pgfpoint{.5\pgfphysicalwidth}{.5\pgfphysicalheight}%
  }%
}

\pgfpagesuselayout{boxed}

\begin{document}
	\maketitle	
\begin{large}
\begin{center}
	\textbf{EXPONENTIAL FUNCTIONS}  
\end{center}
\textbf{Terminology}


\begin{center}
Exponential functions $f$ with base $a$ \\
\noindent $\rightarrow$ Domain = $(-\infty, \infty)$ \\
$\rightarrow$ Range = $(0, \infty)$
\end{center}
\bigbreak \noindent \bigbreak \noindent
\dfn{}{
$f(x) = a^x$ for every x in $R$ where $a > 0$ and $a\neq 1$}
\end{large}
\bigbreak \noindent \bigbreak \noindent
\begin{large}

	\noindent \textbf{Theorem}
\bigbreak
\begin{mdframed}
	\noindent The exponential function $f$ given by $f(x) = a^x$ for $ 0 < a < 1$ or $ a > 1$ is one-to-one. Thus, the following conditions are satisfied for real numbers $x_1$ and $x_2$.
	\begin{enumerate}
		\item if $x_1 \neq x_2$, then $a^{x_1} \neq a^{x_2}$ 
		\item if $a^{x_1} = a^{x_2}$, then $x_1 = x^2$
	\end{enumerate}
\end{mdframed}
\end{large}
\bigbreak \noindent \bigbreak \noindent \bigbreak \noindent \bigbreak \noindent
\begin{large}
	\textbf{Recall: Laws of Exponents} 

\begin{enumerate}
	\item $a^m \cdot a^n=a^{m+n}$ 
	\item $\frac{a^m}{a^n}=a^{m-n}$
	\item $a^{-n}=\frac{1}{a^n}$
	\item $(a \cdot b)^n=a^n \cdot b^n$
	\item $\left(\frac{a}{b}\right)^{-n}=\frac{b^n}{a^n}$
	\item $\left(a^m\right)^n=a^{m \cdot n}$
	\item $a^0=1$
\end{enumerate}
\end{large}
\bigbreak \noindent \bigbreak \noindent
\pagebreak
\begin{mdframed}
\begin{large}
\begin{center}
  
Example 1 Solve the equations. \\

\bigbreak
\noindent a) $9^{\left(x^2\right)}=3^{3 x+2}$ \\
\bigbreak
\noindent b) $\left(\frac{1}{2}\right)^{6-2 x}=2$ \\
\bigbreak
\noindent c) $9^{2 x} \cdot 27^{x^2}=3^{-1}$ \\
\bigbreak
\noindent d) $\left(e^4\right)^x \cdot e^{x^2}=e^{12}$ \\

\end{center}
\end{large}
\end{mdframed}

\pf{Solution to Question 1}

\begin{align*}
	(3x^2)^{x^{2}} = 3^{3x+2} \\
	3^{2x^{2}} = 3^{3x+2} \\ 
	2x^2  = 3x+2 \\
	2x^2-3x-2 = 0 \\
	x = -\frac{1}{2}, x=2
.\end{align*}
\bigbreak
\pf{Soultion to Question 2}

\begin{align*}
	\left(2^{-1}\right)^{6-2 x}=2 \\
	2^{-6+2 x}=2^1 \\
	-6+2x=1 \\
	2x=7 \\
	x = \frac{7}{2}
\end{align*}
\bigbreak
\pf{Solution to Question 3}

\bigbreak \noindent
\begin{center}
  $\left(3^2\right)^{2 x} \cdot\left(3^3\right)^{x^2}=3^{-1}$ \\
  $3^{4 x} \cdot 3^{3 x^2}=3^{-1}$ \\
  $3^{4 x+3 x^2}=3^{-1}$ \\
  $4 x+3 x^2=-1$ \\
  $3 x^2+4 x+1=0$ \\
  $(3 x+1)(x+1)=0$ \\
  $x=-\frac{1}{3},-1$
\end{center}
\bigbreak \noindent
\pf{Soultion to Question 4}

\bigbreak \noindent

$$
e^{4 x} \cdot e^{x^2}=e^{12}
$$
$$
e^{4 x+x^2}=e^{12}
$$
$$
4 x+x^2=12
$$
$$
x^2+4x-12=0
$$ 
$$
(x+6)(x-2)=0
$$
$$
x=-6, 2
$$
\bigbreak \noindent \bigbreak \noindent
\begin{mdframed}
\begin{large}
\begin{center}
	\textbf{Example 2} 
	Sketch the graph of $f$ if
$$
\text { a) } f(x)=-3^{\wedge} x+9
$$
$$
\text { b) } f(x)=2^{\wedge}\{-(x+1)\}
$$

\end{center} 
\end{large} 
\end{mdframed}
\bigbreak

\pf{Soultion to Question 1}

	\begin{center}
		\begin{tikzpicture}
			\begin{axis}[restrict y to domain=-10:10, restrict x to domain=-10:10,samples=200, xmin=-10, xmax=10, ymin=-10, ymax=10, axis lines=middle]
				\addplot[color=red]{-3^x+9};
			\end{axis} 
		\end{tikzpicture}
	\end{center}
\bigbreak \noindent \bigbreak \noindent

\pf{Soultion to Question 2}

	\begin{center}
		\begin{tikzpicture}
			\begin{axis}[restrict y to domain=-10:10, restrict x to domain=-10:10,samples=200, xmin=-10, xmax=10, ymin=-10, ymax=10, axis lines=middle]
				\addplot[color=red]{2^-(x+1)};
			\end{axis} 
		\end{tikzpicture}
	\end{center}
\bigbreak \noindent \bigbreak \noindent

\dfn{number $e$}{
	\begin{center}
	If $n$ is a positive integer, then $\left(1+\frac{1}{n}\right)^n \rightarrow e \approx 2.71828$ as $n \rightarrow \infty$.
	\end{center}
}
\bigbreak \noindent
\nt{The natural exponential function $f$ is defined by $f(x)=e^x$ for every real number $x$.}
\bigbreak \noindent \bigbreak \noindent
\begin{large}
 \textbf{Example 3}
 Graph $f(x) = e^x+4$
\end{large}

\pf{Soultion}

	\begin{center}
		\begin{tikzpicture}
			\begin{axis}[restrict y to domain=-10:10, restrict x to domain=-10:10,samples=200, xmin=-10, xmax=10, ymin=-10, ymax=10, axis lines=middle]
				\addplot[color=red]{e^x+4};
			\end{axis} 
		\end{tikzpicture}
	\end{center}



\end{document}
