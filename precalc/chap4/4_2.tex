\documentclass{report}

\input{~/latex/template/preamble.tex}
\input{~/latex/template/macros.tex}

\title{\Huge{Chapter 4.2 Notes}}
\author{\huge{Matt Warner}}
\date{\huge{}}
\pagestyle{fancy}
\fancyhf{}
\rhead{ONE-TO-ONE FUNCTIONS; INVERSE FUNCTIONS}
\lhead{\leftmark}
\cfoot{\thepage}
% \usepackage[default]{sourcecodepro}
% \usepackage[T1]{fontenc}

\usepackage{pgfplots}
\pgfplotsset{compat=1.18}

\pgfpagesdeclarelayout{boxed}
{
  \edef\pgfpageoptionborder{0pt}
}
{
  \pgfpagesphysicalpageoptions
  {%
    logical pages=1,%
  }
  \pgfpageslogicalpageoptions{1}
  {
    border code=\pgfsetlinewidth{1.5pt}\pgfstroke,%
    border shrink=\pgfpageoptionborder,%
    resized width=.95\pgfphysicalwidth,%
    resized height=.95\pgfphysicalheight,%
    center=\pgfpoint{.5\pgfphysicalwidth}{.5\pgfphysicalheight}%
  }%
}

\pgfpagesuselayout{boxed}

\begin{document}
\maketitle		

\dfn{}{A function $f$ with domain $D$ and range $R$ is a one-to-one function if either of the following equivalent conditions is satisfied.
	\begin{enumerate}
	\item Whenever $a \neq b$ in $D$, then $f(a) \neq f(b) in R.$ 
	\item Whenever $f(a) = f(b) in R$, then $a = b$ in $D$.
	\end{enumerate}
}
\bigbreak \bigbreak
\begin{mdframed}
	\begin{center}
		\textbf{Example 1}
		Determine whether the function $f(x) =2x^3-4$ is one-to-one
	\end{center}
\end{mdframed}
\bigbreak

\pf{\large{Solution For Example 1}}{
	\begin{align*
		f(a) = f(b) \\
		2a^3-4 = 2b^3-4 \\
		2a^3 = 2b^3 \\
		a^3 = b^3 \\
		a = b
	.\end{align*}
Since $a = b$, the function is one-to-one.
}

\bigbreak \bigbreak
\nt{\textbf{Horizontal Line Test} \\ a function $f$ is one-to-one is and only if every horizontal line intersects the graph of $f$ in at most one point.}
\bigbreak
\begin{mdframed}
	\begin{center}
		\textbf{Example 2} 
		Use the Horizontal line test to determine if the following are one-to-one functions.
	\end{center}
\end{mdframed}

\qs{}{
	$$f(x)=x^2-1$$
}
\pf{Solution to Question 1}


	\begin{center}
		\begin{tikzpicture}
			\begin{axis}[restrict y to domain=-10:10, restrict x to domain=-10:10,samples=200, xmin=-10, xmax=10, ymin=-10, ymax=10, axis lines=middle]
				\addplot[color=red]{x^2-1};
			\end{axis} 
		\end{tikzpicture}
	\end{center}
	\textbf{Not one-to-one}
\bigbreak
\qs{}{
	$$f(x) = \frac{2}{3}x+1$$
}
	\begin{center}
		\begin{tikzpicture}
			\begin{axis}[restrict y to domain=-10:10, restrict x to domain=-10:10,samples=200, xmin=-10, xmax=10, ymin=-10, ymax=10, axis lines=middle]
				\addplot[color=red]{2/3(x)+1};
			\end{axis} 
		\end{tikzpicture}
	\end{center}
\textbf{Yes, the graphs shows the function to be one-to-one}
\bigbreak \bigbreak
\nt{
	\begin{enumerate}
		\item A function that is increasing throughout its domain is one-to-one. 
		\item A function that is decreasing throughout its domain is one-to-one.
	\end{enumerate}
}
\bigbreak \bigbreak
\dfn{}{Let $f$ be a one-to-one function with domain $D$ and range $R$. A function $g$ with domain $R$ and range $D$ is the inverse function of $f$, provided the following conditions is true for every $x$ in $D$ and every $y$ in $R$:.
}









\end{document}
