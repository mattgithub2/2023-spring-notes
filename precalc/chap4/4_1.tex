\documentclass{report}

\input{~/latex/template/preamble.tex}
\input{~/latex/template/macros.tex}

\title{\Huge{Chapter 4.1 Notes}}
\author{\huge{Matt Warner}}
\date{\huge{}}
\pagestyle{fancy}
\fancyhf{}
\rhead{SECTION 4.1}
\lhead{\leftmark}
\cfoot{\thepage}
% \usepackage[default]{sourcecodepro}
% \usepackage[T1]{fontenc}

%\usepackage{pgfplots}
%\pgfplotsset{compat=1.18}

\pgfpagesdeclarelayout{boxed}
{
  \edef\pgfpageoptionborder{0pt}
}
{
  \pgfpagesphysicalpageoptions
  {%
    logical pages=1,%
  }
  \pgfpageslogicalpageoptions{1}
  {
    border code=\pgfsetlinewidth{1.5pt}\pgfstroke,%
    border shrink=\pgfpageoptionborder,%
    resized width=.95\pgfphysicalwidth,%
    resized height=.95\pgfphysicalheight,%
    center=\pgfpoint{.5\pgfphysicalwidth}{.5\pgfphysicalheight}%
  }%
}

\pgfpagesuselayout{boxed}

\begin{document}
\maketitle
\begin{Large}
	\noindent \textbf{COMPOSITE FUNCTIONS:}
\end{Large}	
\bigbreak \noindent \bigbreak \noindent
\dfn{}{
Given two functions $f$ and $g$ the composite function, denoted by $f \circ g$ (read as ''$f$ composed with $g$``), is defined by $f \circ g = f(g(x))$
}
\bigbreak
\noindent \nt{The Domain of $f \circ g$ is the set of all numbers x in the domain of $g$ such that $g(x)$ is in the domain of $f$.}
\bigbreak \noindent
\begin{mdframed}
	\begin{center}

		\textbf{Example 1 - Find: a. $f \circ g(4)$ b. $g \circ f(2)$ c. $f \circ f(1)$ d. $g \circ g(0)$}

	\end{center}
\end{mdframed}

\bigbreak
\begin{center}
\noindent {1) $f(x) = \sqrt{x+1}$ and $g(x)=3x$}
\end{center}
\bigbreak
\pf{solution}
\noindent a) $f(g(4)) = \sqrt{13}$
\bigbreak
b) $g(f(2)) = 3\sqrt{3}$
\bigbreak
c) $f(f(1)) = \sqrt{\sqrt{2} +1}$
\bigbreak
d) $g(g(0)) = 0$
\bigbreak
\begin{center}
	2) $ f(x) = | x-2 |$ and $g(x) = \frac{3}{x^2+4}$
  
\end{center}
\bigbreak
\pf{Solution}

a) $ f(g(4)) = \frac{37}{20}$
\bigbreak
b) $g(f(2)) = \frac{3}{4}$
\bigbreak
c) $f(f(1)) = 1$
\bigbreak
d) $g(g(0)) = \frac{48}{73}$
\bigbreak \noindent \bigbreak \noindent
\begin{mdframed}
	\begin{center}
		\textbf{Example 2: Find the domain of the composite function $f \circ g$}
	  
	\end{center}
  
\end{mdframed}
\nt{when finding the domain of f(g(x)), you include the restrictions for g(x) and any new restrictions in $ f \circ g$}
\qs{}{
	$$f(x) = \frac{x}{x+3}$$
	$$g(x) = \frac{2}{x}$$
}
  
\bigbreak
\pf{Solution ($f \circ g$)}

\bigbreak
\textit{domain of $g(x)$: $x \neq{0}$}
\bigbreak
$f \circ g$ : $\frac{\frac{2}{x}}{\frac{2}{x}+3}$ $\rightarrow$ $\frac{2}{2+3x}$
\bigbreak
\textit{domain of $f(g(x))$: $x \neq{-\frac{2}{3}}$}
\bigbreak
\textbf{so,} domain: $\{x\neq{0},x\neq{-\frac{2}{3}}$\}
\bigbreak \bigbreak
\qs{}{
	$$f(x)=x^{2}+4$$ 
	$$g(x)=\sqrt{x-2}$$
}
\pf{solution ($f \circ g$)}

\bigbreak
\textit{Domain of g(x): $ x | x \ge 2$}
\bigbreak
$f(\sqrt{x-2}) = (\sqrt{x-2})^2+4$
\bigbreak
$ x-2+4$ $\rightarrow$ $x+2$ (we can ignore this)
\bigbreak
\textbf{so,} Domain: $ \{x | x \ge 2$\}
\bigbreak \noindent \bigbreak \noindent
\begin{mdframed}
 \begin{center}
	 \textbf{Example 3. Find: i. $f \circ g$ ii. $g \circ f$ iii. $f \circ f$ iv. $g \circ g$}
 \end{center} 
\end{mdframed}
\begin{large}
	\begin{center}
\textbf{For the given functions $f$ and $g$. State the domain of each composite function.} 
	\end{center}
\end{large}
\qs{}{
	$$f(x)=x^2+1$$
	$$g(x)=2x^2+3$$
}
\pf{Solution i. ($f\circ g$)}
\bigbreak

Domain of $g(x)$ $\rightarrow$ $D:\mathbb{R}$
\bigbreak
$f(2x^2+3) = (2x^3+3)^2 + 1$
\bigbreak
$\rightarrow$ $4x^4+12x^2+10$ (No restrictions in domain)
\bigbreak
\textbf{So,} $D:\mathbb{R}$
\bigbreak
\pf{Solution ii. ($g \circ f$)}
\bigbreak

Domain of $f(x)$ $\rightarrow$ $D:\mathbb{R}$
\bigbreak
$g(x^2+1) = 2(x^2+1)^2+3$
\bigbreak
= $2(x^4+2x^2+1)+3$
\bigbreak
= $2x^4+4x^2+5$ (No restrictions in domain)
\bigbreak
\textbf{So,} $D:\mathbb{R}$
\bigbreak
\pf{Solution iii. ($f \circ f$)}
\bigbreak

Domain of $f(x)$ $\rightarrow$ $D:\mathbb{R}$
\bigbreak
$f(x^2+1) = (x^2+1)^2+1$
\bigbreak
= $x^4+2x^2+1$ (No restrictions in Domain)
\bigbreak
\textbf{So,} $D:\mathbb{R}$
\pf{Solution iv. ($ g \circ g$)}
\bigbreak
Domain of $g(x)$ $\rightarrow$ $D:\mathbb{R}$
\bigbreak
$g(2x^2+3) = 2(2x^2+3)^2+3$
\bigbreak
= $8x^4+24x^2+21$ (no domain restrictions)
\bigbreak
\textbf{So,} $D:\mathbb{R}$
\bigbreak \bigbreak
\qs{}{
	$$f(x)=\sqrt{x-2}$$
	$$g(x)=1-2x$$
}
\pf{Solution i. ($f \circ g$)}
\bigbreak
Domain of $g(x)$ $\rightarrow$ $D:\mathbb{R}$
\bigbreak
$f(1-2x) = \sqrt{1-2x-2}$ $\rightarrow$ $\sqrt{-2x-1}$
\bigbreak
$\sqrt{-2x-1} \ge 0$
\bigbreak
$-2x-1 \ge 0$ $\rightarrow$ $-2x \ge 1$
\bigbreak
$x \le -\frac{1}{2}$
\bigbreak
\textbf{So,} $D:\{x | x \le -\frac{1}{2}\}$
\bigbreak
\pf{Solution ii. ($g \circ f$)}
\bigbreak

Domain of $f(x)$ $\rightarrow$ $ D:\{x | x \ge 2\}$
\bigbreak
$g(\sqrt{x-2}) = 1-2\sqrt{x-2}$
\bigbreak
\textbf{So,} $D: \{x | x \ge 2\}$
\bigbreak

	





\end{document}
	

