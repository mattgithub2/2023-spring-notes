\documentclass{report}

\input{~/latex/template/preamble.tex}
\input{~/latex/template/macros.tex}

\title{\Huge{Chapter 3.4 Notes}}
\author{\huge{Matt Warner}}
\pagestyle{fancy}
\fancyhf{}
\rhead{Chapter 3.4 PROPERTIES OF RATIONAL FUNCTIONS}
\lhead{\leftmark}
\cfoot{\thepage}
% \usepackage[default]{sourcecodepro}
% \usepackage[T1]{fontenc}
\usepackage{pgfplots}
\pgfplotsset{compat=1.18}
\pgfpagesdeclarelayout{boxed}
{
  \edef\pgfpageoptionborder{0pt}
}
{
  \pgfpagesphysicalpageoptions
  {%
    logical pages=1,%
  }
  \pgfpageslogicalpageoptions{1}
  {
    border code=\pgfsetlinewidth{1.5pt}\pgfstroke,%
    border shrink=\pgfpageoptionborder,%
    resized width=.95\pgfphysicalwidth,%
    resized height=.95\pgfphysicalheight,%
    center=\pgfpoint{.5\pgfphysicalwidth}{.5\pgfphysicalheight}%
  }%
}

\pgfpagesuselayout{boxed}

\begin{document}
  \maketitle

  \begin{Large}
    \noindent \textbf{Definition}
  \end{Large}

  \bigbreak
    \noindent A rational function is a function of the form $f(x) = \frac{g(x)}{h(x)}$ where $g(x)$ and $h(x)$ are polynomials
    and $h(x)\neq{0}$. 

  \bigbreak \bigbreak
  \noindent \underline{Recall}
   Domain of rational functions is all real numbers except those that make denominator 0.

  \bigbreak \bigbreak
  \noindent \underline{Example 1}
  find the domain of each rational function
  \bigbreak
  \begin{large}
  \noindent a) $f(x) = \frac{x}{(x-6)(2-x)}$
  \end{large}
  \pf{Solution}
  \bigbreak
  $D: \{{x| x\neq{6,2}}\}$
  
  \bigbreak \bigbreak
  \begin{large}
  \noindent b) $k(x) = \frac{x-1}{x^2+3}$
  \end{large}
  \pf{Solution}
  \bigbreak
  $D: \mathbb{R}$
  \bigbreak \bigbreak
  \begin{large}
    \noindent c) $f(x) = \frac{-2(x^2-1)}{(x^2+2x+1)}$
  \end{large}
  \pf{Solution}
  \bigbreak
  $D: \{x|x\neq{-2}\}$
  \bigbreak \bigbreak
  \begin{large}
    \noindent\textbf{Vertical Asymptotes}
    \bigbreak
    \noindent The Vertical Asymptotes of a rational function can be found by setting the denominator to 0 and solve for x. That is, let $h(x)=0$.
  \end{large}
  \bigbreak \bigbreak 
  \begin{large}
    \noindent \textbf{Horizontal Asymptotes} 
    \bigbreak
    $$Let f(x) = \frac{a_nx^n+a_{n-1}x^{n-1}+..._{...}+a_1+a_0}{b_kx^k+b_{k-1}x^{k-1}+..._{...}+b_1+b_0}$$

  \end{large}
  \bigbreak
  \begin{normalsize}
  \begin{enumerate}
    \item if $n<k$, then the x-axis (the line $y=0$) is the Horizontal Asymptote for the graph of $f$.

    i.e. degree of numerator = degree of denominator implies H.A. is $y=0$
  \item if $n=k$, then the line $y=a_n/b_k$ (the ratio of leading coefficients) is the Horizontal Asymptote for the graph of $f$.
  \item if $n>k$, the graph has no Horizontal Asymptote.

  Instead, either $f(x)\to\infty$ or $f(x)\to-\infty$ as $x\to\infty$ or as $x\to-\infty$
  \bigbreak 
  i.e. degree of numerator > degree of denominator implies no H.A.
  \bigbreak
  There is an O.A. (oblique asymptote) is degree numerator = degree denominatorThere is an O.A. (oblique asymptote) is degree numerator = degree denominator $+ 1$

    
  \end{enumerate}
  \end{normalsize}
  \bigbreak \bigbreak
  \begin{large}
  
  \noindent \textbf{Oblique Asymptote}
  \bigbreak 
  \noindent An oblique asymptote for a graph in a line $y=ax+b$, with $a\neq{0}$, such that the graph approaches this line a $x\to\infty$ or as $x\to-\infty$
  \bigbreak
  \noindent \underline{Note:}
  \noindent The degree of the numerator must be one greater than the degree of the denominator. To find the oblique asymptote, do long division (or synthetic division)
  \end{large}
  \bigbreak \bigbreak
  \begin{large}
    \noindent \underline{Example 2}
    Find the vertical, horizontal, and oblique asymptotes, if any, of each rational function
  \end{large} 
    \bigbreak
    \qs{}{
      $$f(x) = \frac{2x+7}{x^2-3x+2}$$
  }

    \pf{Solution for question 1}{
    \begin{align*}
      f(x)=\frac{2x+7}{(x-1)(x-2)}
    \end{align*}
    \indent \textbf{Vertical Asymptote:} x = 1, x = 2
    \bigbreak
    \indent \textbf{Horizontal Asymptotes:} y = 0
    \bigbreak
    \indent \textbf{Oblique Asymptote:} None
  }
  \bigbreak
  \qs{}{
  $$f(x)=\frac{5x^2+3x-1}{4x+1}$$
  }

    \pf{Solution for question 2}
    
      \bigbreak
      \indent \textbf{Vertical Asymptote:} x = $-\frac{1}{4}$
      \bigbreak
      \indent \textbf{Horizontal Asymptotes:} None
      \bigbreak
      \indent \textbf{Oblique Asymptote:} y = $\frac{5}{4}x + \frac{7}{16}$
  \bigbreak
  \qs{}{
    $$f(x)=\frac{6x^2-4x+3}{2x^2+x-1}$$

  }
  \pf{Solution to question 3}{

  \begin{align*}
    f(x)=\frac{6x^2-4x+3}{(x+1)(2x-1)}
  \end{align*}
      \bigbreak
      \indent \textbf{Vertical Asymptote:} x = -1, x = $\frac{1}{2}$
      \bigbreak
      \indent \textbf{Horizontal Asymptotes:} y = 3
      \bigbreak
      \indent \textbf{Oblique Asymptote:} None
  }
  \qs{}{
  $$f(x)=\frac{x^2+x-12}{x^2-9}$$
  }
  \pf{Solution to question 4}{
    \begin{align*}
      f(x)=\frac{(x-3)(x+4)}{(x+3)(x-3)} \\
      f(x)=\frac{x+4}{x+3}
    \end{align*}
      \bigbreak
      \indent \textbf{Vertical Asymptote:} x = - 3
      \bigbreak
      \indent \textbf{Horizontal Asymptotes:} y = 1
      \bigbreak
      \indent \textbf{Oblique Asymptote:} None
  }
  \bigbreak \bigbreak \bigbreak
  \begin{large}
    \noindent \underline{Example 3}
    Graph each rational function using transformations 
  \end{large}
  \bigbreak
  \qs{}{
  $$f(x)=3+\frac{1}{x^2}$$
  }
  \pf{Solution to question 5}

  Shift vertically by a factor of 3
  \bigbreak
 \begin{center}
  \begin{tikzpicture}
    \begin{axis}[restrict y to domain=-10:10,samples=200, xmin=-10, xmax=10, ymin=-10, ymax=10, axis lines=middle]
      \addplot[color=red]{3+(1/x^2)};
      \addplot[color=blue, dashed]{3};
    \end{axis} 
  \end{tikzpicture}
  \end{center}
  
  \qs{}{
    $$f(x)=\frac{2}{(x+3)^2}$$
  }
  \bigbreak \bigbreak \bigbreak
  \pf{Solution to question 6}{


  \begin{align*}
   2 * \frac{1}{(x+3)^2}
  \end{align*}
  \bigbreak
  \begin{center}
    1. Shift up 2
    \bigbreak
    2. Shift left 3 
  \bigbreak
  \end{center}
    \begin{center}
      \begin{tikzpicture}
        \begin{axis}[restrict y to domain=-10:10, restrict x to domain=-10:10,samples=500, xmin=-10, xmax=10, ymin=-10, ymax=10, axis lines=middle]
          \addplot[color=red]{2/(x+3)^2};
        \end{axis} 
      \end{tikzpicture}
    \end{center}
  
  } 




\end{document}
