\documentclass{report}

\input{~/latex/template/preamble.tex}
\input{~/latex/template/macros.tex}

\title{\Huge{Section 3.1 Notes}}
\author{\huge{Matt Warner}}
\pagestyle{fancy}
\fancyhf{}
\rhead{{3.1 Polynomial functions and models}}
\lhead{\leftmark}
\cfoot{\thepage}
% \usepackage[default]{sourcecodepro}
% \usepackage[T1]{fontenc}
\usepackage{pgfplots}
\pgfplotsset{compat=1.18}
\pgfpagesdeclarelayout{boxed}
{
  \edef\pgfpageoptionborder{0pt}
}
{
  \pgfpagesphysicalpageoptions
  {%
    logical pages=1,%
  }
  \pgfpageslogicalpageoptions{1}
  {
    border code=\pgfsetlinewidth{1.5pt}\pgfstroke,%
    border shrink=\pgfpageoptionborder,%
    resized width=.95\pgfphysicalwidth,%
    resized height=.95\pgfphysicalheight,%
    center=\pgfpoint{.5\pgfphysicalwidth}{.5\pgfphysicalheight}%
  }%
}

\pgfpagesuselayout{boxed}

\begin{document}
  \maketitle

  \begin{Large}
  \noindent \textbf{Definition:}

  \end{Large}  

  \bigbreak \noindent
  A polynomial function is a function of form $f(x) = a_nx^n + a_{n-1} + a_1x +a_0$ 
  \bigbreak \noindent
  where $a_n, a_{n-1},... ... ..., a_1, a_0$ are real numbers and $n$ is a nonnegative integer. 
  The degree of a polynomial is the highest degree of its terms

  \noindent \underline{Recall:}{
    The domain of a polynomial function is all real numbers.
  }

  \bigbreak
  \noindent \underline{Example 1}{
    Determine whether the functions are polynomials or not.
    For those who are state the degree of the polynomia.
    For those who are not state why not.
  }
  \bigbreak \noindent
  a) $f(x) = x(x+5)$
  \pf{Answer}{}
  \bigbreak
   \textbf{factored form }= $f(x) = x^2 + 5x$
   \bigbreak
   is a polynomial
  \bigbreak 
  \textit{Degree: 2}
  
  \bigbreak \bigbreak \noindent
  b) $f(x) = \sqrt{x}(\sqrt{x} - 2)$
  \pf{Answer}
  \bigbreak
  
  Not a polynomial, not an integer exponent
  \bigbreak \bigbreak \noindent
  c) $f(x) = \frac{x^2 - 5x}{x^3}$
  \pf{Answer}
  
  Can be re-written as: $\frac{x^2}{x^3}$ - $\frac{5x}{x^3}$
  \bigbreak
  =  $x^{-1} - 5x^{-2}$
  \bigbreak
  since leading coeficient has negative exponent, it is \textbf{not} a polynomial

  \bigbreak \bigbreak \noindent
  d) $f(x) = -3x^2(x+5)^3$
  \pf{Answer}

  = $-3x^2(x^3+...)$
  \bigbreak 
  This is a polynomial
  \bigbreak
  \textit{degree: 5}

  \bigbreak \bigbreak \noindent
  \begin{Large}

  \noindent  \underline{Power Functions}

  \end{Large}
  \bigbreak \noindent
  A \textbf{Power Function} of degree $n$ is a \textbf{monomial} of the form $f(x) = ax^n$
  where $a$ is a real number, $a \neq{0}$, and $n$ is a \textbf{positive integer}
  \bigbreak \noindent
  if $n$ is an \textbf{even integer}, then the following are true about the power functions:
  \begin{enumerate}
  \item The graph of the function is symmetric over the \textbf{y-axis}
  \item $D$ = all real numbers; $R$ = $[0,\infty]$
  \item The graph resembles the graph of $y = x^2$
  \end{enumerate}
  if $n$ is an \textbf{odd integer}, then the following are true about the power function:
  \begin{enumerate}
    \item The graph of the function is symmetric over the origin
    \item $D$ = all real numbers; $R$ = all real numbers
    \item The graph resembles the graph of $y = x^3$
  \end{enumerate}
  \bigbreak \bigbreak

  \noindent \underline{Example 2}{
  Use Transformations to graph each function
}
  \bigbreak
  \noindent a) $f(x) = -x^4 + 3$

  \begin{tikzpicture}

    \begin{axis}[xmin=-10, xmax=10, ymin=-10, ymax=10, axis lines=middle]

      \addplot[color=blue]{x^4};
      \addplot[color=red, dashed]{-x^4 + 3};

    \end{axis}

  \end{tikzpicture}
  \pf{Steps}

  Parent Function: $y = x^4$ (Blue line)

  negative sign infront of the leading coeficient represents: \textbf{Reflect over the x-axis}

  constant is $+3$ which reprents: \textbf{shift up 3 units}
  \bigbreak \bigbreak \noindent
  b) $f(x) = \frac{1}{2}(x-1)^5 - 2$

  \begin{tikzpicture}

    \begin{axis}[xmin=-10, xmax=10, ymin=-10, ymax=10, axis lines=middle, restrict y to domain=-10:10, restrict x to domain=-10:10]
      \addplot[color=blue]{x^3};
      \addplot[color=red, samples=200, dashed]{1/2(x-1)^5-2};
    \end{axis}

  \end{tikzpicture}
  \pf{Steps}

  Parent Function: $y = x^5$ (Blue line)

  1/2 represents: \textbf{compress by 1/2} (multiply y by 1/2)

  -1 represents: \textbf{Shift to the right 1}
  
  -2 represents: \textbf{shift down 2}

  \bigbreak \bigbreak

  \begin{Large}
    \noindent \textbf{Definition}
  \end{Large} 
  
  \bigbreak \noindent
  if $f$ is a function and $r$ is a real number for which $f(r) = 0$, then $r$ is called a real zero of $f$.

  \noindent if $(x-r)^m$ is a factor of a polynomial $f$ and $(x-r)^{m+1}$ is \textbf{not} a factor, then $r$ is called a zero of multiplicity $m$ of $f$

  \bigbreak

  \noindent That is:
  \begin{enumerate}

  \item $r$ is a real zero of a polynomial function $f$.
  \item $r$ is an x-intercept of the graph of $f$.
  \item $x-r$ is a factor of $f$.

  \end{enumerate}
  
  \bigbreak \bigbreak
  \noindent \underline{Example 3}
  Form a polynomial whose real zeros and degree are given.
  \bigbreak
  \noindent a) zeros: -2,2,3; degree: 3
  \bigbreak
  $p(x) = (x+2)(x-2)(x-3)$
  \bigbreak
  = $p(x) = x^3-3x^2-4x+12$
  \bigbreak \bigbreak
  \noindent b) zeros: -3,-1,2,5 degree: 4
  \bigbreak 
  $p(x) = (x+3)(x+1)(x-2)(x+5)$
  
  = $p(x) = x^4-3x^3-15x^2+19x+30$ 
  \bigbreak \bigbreak \noindent
  \begin{Large}

    \noindent \textbf{Multiplicity and Turning Points}
    
  \end{Large}
  \bigbreak 
  \noindent If the zero is of \textbf{even multiplicity}; the graph of $f$ \textbf{touches} the x-axis at that zero
  
  \noindent If the zero is of \textbf{odd multiplicity}; the graph of $f$ \textbf{crosses} the x-axis at that zero
  \bigbreak \bigbreak  

  \begin{Large}
  \noindent \underline{Definition}
  \end{Large}
  
  \bigbreak \noindent
  The points at which a graph changes direction are called \textbf{turning points} (each turning point yields a local maximum or local maximum)
  \bigbreak
  \noindent if $f$ is a polynomial of degree $n$, then $f$ has at most $n-1$ turning points.
  
  \noindent i.e. \textbf{maximum number of turning points = degree -1}
  \bigbreak \noindent
  For large values of $x$, either positive or negative, the graph of the polynomial

  \noindent $f(x) = a_nx^n + a_{n-1} + ..._... + a_nx + a + 0$   

  \noindent $y = a_nx^n$. The behavior of the graph of a function for large values of $x$, either positive and negative, is referred to as its end behavior
  \bigbreak \noindent \underline{Example 4} For each polynomial function
  \begin{enumerate}

    \item List each real zero and its multiplicity
    \item Find the x-axis and y-intercepts
    \item Determine whether the graph crosses or touches the x-axis at each x-intercept 
    \item Determine the maximum number of turning points on the graph
    \item Determine the end behavior of the function
    \item Sketch the graph of the polynomial

  \end{enumerate}
  \bigbreak
  \noindent a) $f(x) =(x-\frac{1}{3})^2(x-1)^3$
  \pf{Solution}

    Zeros: $\frac{1}{3}, 1$
    \bigbreak
    Multiplicity = 2, 3
    \bigbreak
   x-intercepts: $\frac{1}{3}$ (touches), 1 (crosses) 
    \bigbreak
   y-intercepts = $-\frac{1}{9}$
    \bigbreak
    Max number of turning points: 4
    \bigbreak
    end-behavior: as $x\to\infty$,  $f(x)\to\infty$
    as $x\to-\infty$, $f(x)\to-\infty$
    \bigbreak \bigbreak \noindent
    b) $f(x) = -2(x^2+1)^3$
    \pf{Solution}

      Zeros: None
      \bigbreak 
      Multiplicity: None
      \bigbreak
      x-intercepts: None
      \bigbreak
      y-intercepts: -2
      \bigbreak
      Max number of turning points: 5
      \bigbreak
      end-behavior: as $x\to\infty$, $f(x)\to-\infty$ as $x\to-\infty$, $f(x)\to
      -\infty$
      \bigbreak \bigbreak
      \noindent c) $f(x) = 5x(x+3)^3$
      \pf{Solution}

      Zeros: 0, -3
      \bigbreak
      Multiplicity: 3, 1
      \bigbreak
      x-intercepts: 0,-3 
      \bigbreak
      y-intercepts: (0,0)
      \bigbreak
      Max number of turning points: 3
      \bigbreak
      \begin{center}
       end-behavior:$$\lim_{x\to\infty} f(x)=\infty, \lim_{x\to-\infty} f(x) = -\infty$$
      \end{center}

    \bigbreak \bigbreak
    \noindent d) $f(x) = -x^2(x^2-1)(x+1)^3$
    \pf{Solution}

    Zeros: 0, -1,1
    \bigbreak
    Multiplicity: 2, 1, 4
    \bigbreak
    x-intercepts: 0,-1,1
    \bigbreak
    y-intercepts: 0
    \bigbreak
    Max number of turning points: 6
    \bigbreak
    \begin{center}
      End behavior: $$\lim_{x\to\infty} f(x) = -\infty, \lim_{x\to-\infty} f(x) = \infty$$
      
    \end{center}
      

      


  





\end{document}

