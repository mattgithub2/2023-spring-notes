\documentclass{report}

\input{~/latex/template/preamble.tex}
\input{~/latex/template/macros.tex}

\title{\Huge{Chapter 11.3 Notes}}
\author{\huge{Matt Warner}}
\date{\huge{}}
\pagestyle{fancy}
\fancyhf{}
\rhead{}
\lhead{\leftmark}
\cfoot{\thepage}
% \usepackage[default]{sourcecodepro}
% \usepackage[T1]{fontenc}

\usepackage{tikz}
\usepackage{pgfplots}
\pgfplotsset{compat=1.18}

\pgfpagesdeclarelayout{boxed}
{
  \edef\pgfpageoptionborder{0pt}
}
{
  \pgfpagesphysicalpageoptions
  {%
    logical pages=1,%
  }
  \pgfpageslogicalpageoptions{1}
  {
    border code=\pgfsetlinewidth{1.5pt}\pgfstroke,%
    border shrink=\pgfpageoptionborder,%
    resized width=.95\pgfphysicalwidth,%
    resized height=.95\pgfphysicalheight,%
    center=\pgfpoint{.5\pgfphysicalwidth}{.5\pgfphysicalheight}%
  }%
}

\pgfpagesuselayout{boxed}

\begin{document}
	\maketitle
\begin{LARGE}
 \begin{center}
	 \textbf{GEOMETRIC SEQUENCE} 
 \end{center} 
\end{LARGE}
\bigbreak \noindent \bigbreak \noindent

\noindent \begin{large}
 Determine Whether a Sequence is Geometric 
	\bigbreak \noindent 
\thmcon{
	\textbf{\underline{Defintion}}
	\vspace{2mm}

	The formula for a Geometric Sequence is as follows:
	\bigbreak \noindent
	 $$ a_1 = a \hspace{10mm} a_n = ra_{n-1}$$
	\vspace{3mm}

	where $r$ = the common ratio
	\vspace{6mm}

	Formula for $r$
	$$r = \frac{a_n}{a_{n-1}}$$
	\bigbreak \noindent
	Formula for sum of a geometric sequence:

	$$
	\text { sum: } s_n=\frac{a_1\left(1-r^n\right)}{(1-r)}
	$$
	\bigbreak \noindent
	Formula for $n$th term

	$$
	\text { nth term: } a_n=a_1 r^{(n-1)}
	$$

}
\end{large}
\bigbreak \noindent \bigbreak \noindent
\begin{large}
 \begin{center}
	 \textbf{Examples} 
 \end{center} 
\end{large}

\ex{Show that the sequence is geometric. List the first term and the common ratio.}{
	\vspace{3mm}

	a) $2,8,32,128$
}
	\bigbreak \noindent

	Using the formula  $$ r = \frac{a_n}{a_{n-1}}$$
	\vspace{3mm}
	
	We find that $$r = \frac{8}{2} = 4$$
	\vspace{3mm}

	So,
	$$a_1 = 2 \hspace{5mm} r = 4$$
\pagebreak
\ex{Show that the sequence is geometric. List the first term and the common ratio.}{
	\vspace{3mm}

b) $ \{s_n\} = \{3^{n+1}\}$
}
\bigbreak \noindent

Given that $s_n = 3^{n+1}$ we can use the formula 
\vspace{2mm}

$$r = \frac{a_n}{a_{n-1}}$$
\vspace{2mm}

So, 

$$ \frac{3^{n+1}}{3^{n-1+1}} = \frac{3^{n+1}}{3^n}$$

\nt{Recall rules of exponents:
	\vspace{3mm}

When you have the same base and you are dividing, you subtract the exponents}
\bigbreak 
So,

$$\frac{3^{n+1}}{3^n} = 3^{n+1 -n}$$
\vspace{3mm}

After simplifing we get

$$r = 3$$
\vspace{3mm}

Therefore,

$$s_1 = 9 \hspace{6mm} r = 3$$
\bigbreak \noindent \bigbreak \noindent

\ex{Show that the sequence is geometric. List the first term and common ratio.}{
	\vspace{3mm}

(c) $\left\{t_n\right\}=\left\{3(2)^n\right\}$
}
\bigbreak \noindent

Given that $s_n = 3(2)^2$, we can use the formula

$$r= \frac{s_n}{s_{n-1}}$$
\vspace{3mm}

So,

$$ r = \frac{3\left(2^n\right)}{3\left(2^{n-1}\right)} = 2^{n - (n-1)}$$

Therefore,

$$ s_1 = 6 \hspace{6mm} r = 2$$
\pagebreak
\ex{Find the ninth term of the geometric sequence and find a recursive formula for the sequence}{
	\vspace{3mm}

	$3,2,\frac{4}{3}, \frac{8}{9}$
}
\bigbreak \noindent

Given that $s_1 = 3$ we need to find $r$,

$$ r = \frac{s_n}{s_{n-1}} = \frac{2}{3} = r$$ 

Now that we have $s_1$ and $r$, we can use the formula 

$$s_n = s_1(r)^{n-1}$$

So, 

$$a_9 = 3\left(\frac{2}{3}\right)^8$$
\vspace{3mm}

Recursive formula

$$a_n = r \cdot a_{n-1}$$

So,

$$a_n = \frac{2}{3} \cdot a_{n-1}$$
\bigbreak \noindent \bigbreak \noindent \bigbreak \noindent
\begin{large}
Sum of $n$ Terms of a Geometric Sequence
\end{large}
\bigbreak \noindent
Let $\{a_n\}$ be a geometric sequence with first term $a$ and common ratio $r$, where $r \neq 0, r \neq 1$. The sum $s_n$ of the first terms of $\{a_ n\}$ is

$$S_n=a \cdot \frac{1-r^n}{1-r}, r \neq 0,1$$
\bigbreak \noindent
\ex{Find the sum of the first $n$ terms of the sequence}{
	\vspace{3mm}

	$\{3^n\}$
}
$$ r = \frac{3^n}{3^{n-1}}$$
$$ r = 3^{n-(n-1)} = 3$$

$$ r = 3 \hspace{6mm} s_1 = 3$$

So by using the formula

$$
s_n=\frac{a_1\left(1-r^n\right)}{(1-r)}
$$

$$\frac{3(1-3^n)}{-2} = \boxed{-\frac{3}{2}\left(1-3^n\right)}$$

\pagebreak
\noindent \begin{large}
	\textbf{Determine Whether a Geometric Series Converges or Diverges} 
\end{large}
\bigbreak \noindent
\thmcon{
	\textbf{\underline{Defintion}}
	\vspace{3mm}

	An infinite sum of the form
$$
a+a r+a r^2+\cdots+a r^{n-1}+
$$
with the first term a and a common ratio $r$, is called an infinite geometric series and is denoted by
$$
\sum_{k=1}^{\infty} a r^{k-1}
$$
\bigbreak \noindent \bigbreak \noindent
	{\textbf{Sum of an Infinite Geometric Series}}
	\vspace{4mm}

	If $|r|<1$, the sum of the infinite geometric series $\sum_{k=1}^{\infty} a r^{k-1}$ is
$$
\sum_{k=1}^{\infty} a r^{k-1}=\frac{a}{1-r}
$$
}
\bigbreak \noindent \bigbreak \noindent
\ex{Find the sum of the geometric series}{
	\vspace{3mm}

	$1+\frac{1}{3}+\frac{1}{9}+\ldots$
}
$$ a = 1 \hspace{10mm} r = \frac{\frac{1}{3}}{1} = \frac{1}{3}$$

Using the formula

$$\frac{a}{1-r}$$

$$\frac{1}{1-\frac{1}{3}}$$
\vspace{.5mm}

So,

$$\text{sum}= \frac{3}{2}$$
\bigbreak \noindent \bigbreak \noindent
\ex{Find the fraction representation of the repeating decimal}{
	\vspace{3mm}

	$0.8888\ldots$
}
$$
0.8+0.08+0.008+\cdots
$$
So,

$$
r = \frac{0.08}{0.8}=0.1
$$

$$ s_1 = 0.8 \hspace{6mm} r = 0.1$$
$$= \frac{8}{9}$$
\pagebreak

\end{document}
